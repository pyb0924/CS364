\documentclass[UTF8]{ctexart}
\usepackage{graphicx}
\usepackage{float}
\usepackage{geometry}
\usepackage{fancyhdr}
\usepackage{lastpage}
\geometry{left=2.54cm,right=2.54cm,top=3.18cm,bottom=3.18cm}%页边距
\pagestyle{fancy}
\lhead{\includegraphics[scale=1]{sjtu-logo-red.pdf}}  
\rhead{Intel与AMD处理器发展历程报告} 
\cfoot{第 \thepage\ 页\ 共 \pageref{LastPage} 页} 

\begin{document}

\begin{titlepage}
    \begin{center}
        \includegraphics[width=0.8\textwidth]{sjtu-name-blue.pdf}\\[1cm]
        \textsc{\Huge \bfseries 课程报告}\\[1.5cm]
        \includegraphics[width=0.3\textwidth]{sjtu-badge-blue.pdf}\\[0.5cm]

        \Huge \bfseries{Intel与AMD处理器发展历程报告}\\[1cm]
        \Large \bfseries{518021910971 裴奕博}
    \end{center}
\end{titlepage}
\section{CPU发展概述}
1947年12月,由美国贝尔实验室的肖克利、巴丁和布拉顿组成的研究小组,发明了晶体管。这种新的材料工艺相比之前的真空电子管,体积小巧、无需预热、耗能极低,很快取代了电子管成为了新一代电子电路的首选。在随后的几十年间,伴随着集成电路的发明,由这种材料制成的电子电路规模越来越大。从小规模、中规模集成电路到大规模、超大规模集成电路。

随着人类对计算机计算能力和便携性的要求不断提升,人们提出了“微型计算机”的概念,要实现这一点,首当其冲的就是将计算机的中央处理单元小型化。1971年,Intel公司制造出了第一个商用微处理器即4004,也宣告了第四代计算机时代的来临。从1971年至今的近50年间,随着个人计算机(PC)的成熟、发展和普及,作为计算机核心的CPU也得以迅猛发展。两家“本是同根生”的半导体公司,Intel和AMD,在这几十年间共同促成了CPU技术的不断提升,时至今日也是市面上处理器的最主流选择。本报告即梳理了从1971年至今,两家公司系列处理器的发展历程。
\section{Intel系列处理器发展历程}
\subsection{1968-1978 Intel公司的创立与第一个处理器的诞生}
\subsection{1978-1994 8086系列处理器与x86架构的诞生}
\subsection{1994-2006 奔腾(Pentium)时代}
\subsection{2006-至今 酷睿(Core)的又一次辉煌}

\section{AMD系列处理器发展历程}
\subsection{1969-1996 AMD公司的创立和Intel的代工厂}
\subsection{1996-1999 K5和K6:自研CPU的初尝试}
\subsection{1999-2006 K7到Athlon 64X2 AMD的崛起和辉煌}
\subsection{2006-2017 AMD失落的十年}
\subsection{2017-至今 锐龙(Ryzen)架构与AMD的重生}
%正文
\end{document}
